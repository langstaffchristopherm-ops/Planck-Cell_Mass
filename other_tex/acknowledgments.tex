This book is built on the work of countless others. I have not created the pieces myself—I have only arranged them in a way that made sense to me. The real credit belongs to the physicists, mathematicians, and thinkers who discovered, tested, and refined the ideas that form this foundation. I am deeply grateful to those who developed the concepts of entropy, thermodynamics, relativity, quantum mechanics, and information theory—the puzzle pieces this thesis tries to weave into a unifying principle.

I am also thankful for the less visible but equally vital sources of strength that make research possible. The encouragement of peers and mentors, the courage it takes to share an idea, and the faith and love of family all create the conditions in which curiosity can flourish. In those moments of support, difficult work becomes sustainable and creative risks become possible.

I’m grateful, too, for the broader culture of science—for the stories, books, lectures, films, and conversations that circulate ideas. These expressions not only share knowledge but also inspire new connections, allowing interpretations to thrive in ways no single discipline could achieve alone. They are a reminder that science is not only equations on a page but also a human story of exploration and meaning.

I also wish to thank the wider community of researchers whose publications, lectures, and discussions made these ideas accessible. Any originality here lies only in nudging familiar pieces into a new arrangement; the substance belongs to those who created the pieces in the first place.

Finally, I’m grateful for readers willing to explore speculative connections and for the spirit of curiosity that sustains physics itself. Thank you—for leaving a trail of insight to follow. Without your efforts, there would be nothing here to assemble.
